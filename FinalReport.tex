%%%%%%%%%%%%%%%%%%%%%%%%%%%%%%%%%%%%%%%%%%%%%%%%%%%%%%%%%%%%%%%%%%%%%%%%%%%%
% Project Title : Parallel Methods in Linear Algebra
% Author        : Robert Pellegrin
% Date          : April 16, 2025
% Course        : MATH306 - Linear Algebra
% Instructor    : Dr. Spickler
% Description   : This document is the final report for the course project.
%                 It includes abstract, background, methods, results,
%                 and future work in standard research paper format.
% 
%%%%%%%%%%%%%%%%%%%%%%%%%%%%%%%%%%%%%%%%%%%%%%%%%%%%%%%%%%%%%%%%%%%%%%%%%%%%

\documentclass[12pt]{article}
\usepackage[margin=1in]{geometry}
\usepackage{setspace}
\usepackage{titlesec}
\usepackage{graphicx} % For images
\usepackage{times} % For Times font
\usepackage{hyperref}
\usepackage{fancyhdr}
\usepackage{caption}
\usepackage{float}


% Header/Footer
\pagestyle{fancy}
\fancyhf{}
\rhead{\thepage}
\lhead{Final Project Report}

% Title formatting
\titleformat{\section}{\normalfont\Large\bfseries}{\thesection}{1em}{}
\titleformat{\subsection}{\normalfont\large\bfseries}{\thesubsection}{1em}{}

\begin{document}

% Title Page
\begin{titlepage}
    \centering
    \vspace*{2in}
    {\Huge\bfseries Parallel Methods in Linear Algebra\par}
    \vspace{1in}
    {\Large Robert Pellegrin\par}
    \vspace{0.5in}
    {\large MATH 306\\ Dr.\ Spickler \\ \today\par}
    \vfill
\end{titlepage}

% Abstract
\newpage
\section*{Abstract}
This project explores the performance of serial and parallel implementations of two core linear algebra operations: \textbf{matrix multiplication}
and \textbf{LU decomposition}. These algorithms were implemented in both C++ and Rust, allowing for a comparison of not only parallel speedup but
also language-level performance differences. The matrix multiplication was done using the standard three-loop algorithm, while LU decomposition
was implemented with partial pivoting and rounding of very small values to zero to reduce numerical errors.

\parskip=1em

Parallel versions were created using OpenMP in C++ and the Rayon library in Rust, both of which enable shared-memory parallelism with minimal
code changes. Performance testing was conducted on square matrices of increasing sizes, with execution times recorded and averaged across
multiple runs to ensure consistency. The results were then analyzed to evaluate the benefits of parallelism and to compare how each language
handles high-performance numerical computing tasks.

\parskip=1em


By benchmarking both serial and parallel variants in each language, this project aims to provide insights into the trade-offs between ease of
implementation, execution speed, and scalability when working with computationally intensive linear algebra algorithms.

% Begin main content
\section{Background and Motivation}
Linear algebra is incredibly important in computer science and related fields, from graphics and machine learning to scientific simulations.
Two common but computationally intensive tasks in linear algebra are matrix multiplication and LU decomposition. Matrix multiplication is fundamental
in areas like computer graphics, graph theory, and neural networks, while LU decomposition helps us solve systems of equations
efficiently---something critical in science, engineering, and computing.

\vspace{1em}

Even though these operations are conceptually straightforward, performing them on large matrices can become extremely slow when using a simple,
one-step-at-a-time (serial) approach. As the amount of data gets bigger, serial implementations of these algorithms can become too slow to remain practical. This is
where parallel computing becomes interesting---by running computations simultaneously on a multi-core CPU, we can potentially
achieve big improvements in speed.

\vspace{1em}

This project focuses on implementing and comparing serial and parallel versions of matrix multiplication and LU decomposition. Specifically,
I'm using two popular programming languages: C++ and Rust. C++ has long been favored for performance-sensitive tasks because it offers powerful
tools for optimization. Rust, though newer, has gained popularity due to its built-in safety features and strong performance, sometimes outperforming
older languages in efficiency and speed.

\parskip=1em

By writing and benchmarking serial and parallel versions of these algorithms in both languages, I want to answer two key questions:

\begin{itemize}
    \item How much faster are parallel versions of these algorithms compared to their serial counterparts?
    \item Does the choice of programming language (C++ versus Rust) significantly impact the performance?
\end{itemize}

This comparison is useful because, as computing increasingly moves toward multi-core processors, it's essential to know which languages and methods can
best take advantage of this hardware. The results could help guide decisions in future projects, especially when performance is critical.


\section{Methods \& Tools}

This project involved implementing two fundamental linear algebra algorithms---\textbf{matrix multiplication} and \textbf{LU decomposition}---in both
serial and parallel forms, using two different programming languages: \textbf{C++} and \textbf{Rust}. The goal was to explore not only how parallelism
improves performance, but also how the choice of language affects execution time and overall efficiency.

\subsection*{Matrix Multiplication}

For matrix multiplication, I used the standard algorithm involving three nested loops. Given two matrices, \( A \) and \( B \), the resulting matrix
\( C \) is calculated such that each element \( C_{ij} \) is the sum of products of the elements from the corresponding row of \( A \) and column of
\( B \):

\[
    C_{ij} = \sum_{k=1}^{n} A_{ik} \times B_{kj}
\]

This method is simple and easy to implement, but it can become computationally expensive for large matrices. To improve performance, I implemented a
parallel version where the outer loop (over rows of the result matrix) is divided across threads. In C++, this was done using \textbf{OpenMP}, which
makes it relatively straightforward to parallelize loops using \texttt{\#pragma omp parallel for}. In Rust, I used the \textbf{Rayon} library, which
provides an easy-to-use \texttt{par\_iter()} API to parallelize iterators safely and efficiently.

Each implementation was tested with square matrices of varying sizes (e.g., $100 \times 100$, 500$ \times $500, 1000$ \times $1000, etc.), and timing was
recorded to observe how execution time scaled with matrix size and how the parallel versions compared to the serial ones.

\subsection*{LU Decomposition}

LU decomposition is used to factor a matrix \( A \) into the product of a lower triangular matrix \( L \) and an upper triangular matrix \( U \), such
that \( A = LU \). I implemented this algorithm using \textbf{partial pivoting}, which helps maintain numerical stability by selecting the largest available
pivot element and swapping rows when needed. This avoids dividing by very small numbers, which can cause large rounding errors.

To further reduce rounding issues, I added a small threshold (e.g., \(1 \times 10^{-10}\)) below which values are rounded to zero. This helps eliminate nonsense
introduced by floating-point arithmetic when working with real numbers in the decomposition.

Parallelizing LU decomposition is more complex than matrix multiplication due to dependencies between rows during the elimination phase. In the parallel version,
I parallelized the row elimination steps after the pivot has been selected. Again, OpenMP was used in C++ to parallelize the inner loops, while in Rust, Rayon's
parallel iterators were used to process multiple rows in parallel. Care was taken to ensure synchronization around shared data to avoid race conditions,
especially when updating pivot rows and performing row swaps.

\subsection*{Languages \& Libraries}

I chose \textbf{C++} for its long-standing reputation in high-performance computing, and \textbf{Rust} for its modern design, strong memory safety guarantees,
and native support for concurrency. The tools and libraries used include:

\begin{itemize}
    \item \textbf{C++}
          \begin{itemize}
              \item GCC compiler with optimization flags (\texttt{-O2}, \texttt{-fopenmp})
              \item OpenMP for parallelism
              \item \texttt{chrono} library for timing
          \end{itemize}

    \item \textbf{Rust}
          \begin{itemize}
              \item Stable Rust compiler (\texttt{cargo})
              \item Rayon crate for data-parallelism
              \item \texttt{std::time::Instant} for timing
          \end{itemize}
\end{itemize}

Both versions of the code were tested on the same machine to ensure consistent comparisons. I also wrote basic matrix generators and along with various
sanity-checking functions to verify the correctness of the algorithms by comparing outputs from serial and parallel versions.

\subsection*{Performance Measurement}

Execution time was the primary metric used to compare the performance of different implementations. Each program was run multiple times, and the average
runtime was recorded to minimize variance caused by system processes or hardware interruptions. Results were organized in tables and plotted to visualize
the differences between serial and parallel execution, as well as between C++ and Rust implementations.

To help automate the benchmarking process, I wrote a Bash script that incrementally increased the matrix size passed to each program. This script looped
through a range of sizes (e.g., from 100 to 10000, increasing by 100 each time), called the appropriate executable, and captured the runtime output.
Automating this step not only ensured consistency across runs but also made it easier to test a broad range of input sizes without manually rerunning
each test.


\section{Results}
Discuss findings, show some figures, tables, and/or code snippets.
Explain results in context stated goals.
% Code for figures. Graphs of timings may go here.
% \newpage
% \section*{Figures}
% \begin{figure}[H]
%     \centering
%     \includegraphics[width=0.8\textwidth]{figure1.png}
%     \caption{Figure caption here}
%     \label{fig:figure1}
% \end{figure}


\section{Summary and Future Work}
While this project focused on matrix multiplication and LU decomposition, there are many other linear algebra algorithms that could benefit significantly
from parallelization. Expanding the scope of this work to include additional algorithms would offer a more comprehensive understanding of where parallel
computing provides the most value in scientific and mathematical computing.

\vspace{1em}


One natural extension would be to explore \textbf{QR decomposition} and \textbf{Cholesky decomposition}, both of which are used in numerical analysis,
machine learning, and solving systems of equations. These algorithms involve matrix transformations that are computationally intensive and contain
steps that could be parallelized similarly to LU decomposition. Another candidate is the \textbf{Singular Value Decomposition (SVD)}, which is widely
used in data science and dimensionality reduction. Although more complex, it presents a great opportunity for parallel optimization due to the large number
of operations involved.

\vspace{1em}

Beyond algorithm selection, there are also more advanced parallelization techniques that could be explored. In this project, shared-memory
parallelism was used via OpenMP in C++ and the Rayon library in Rust. However, these approaches are limited by the number of CPU cores available.
To push performance further, especially for large matrix sizes, \textbf{GPU acceleration} using technologies like \textbf{NVIDIA CUDA} could be
highly effective. GPUs are well-suited for linear algebra due to their high number of processing cores and fast context switching, allowing massive
numbers of computations to run in parallel.

\vspace{1em}

In the Rust ecosystem, GPU programming is still emerging, but there are crates like \texttt{cust} (a CUDA wrapper) that can be used to write Rust
programs that run directly on the GPU. Exploring how GPU-based matrix operations in Rust compare to traditional CPU-based ones would be a valuable
next step. Similarly, \textbf{SIMD} (Single Instruction, Multiple Data) is another performance optimization strategy worth considering. SIMD allows
a single instruction to operate on multiple data points at once, which is perfect for vector and matrix computations. Rust has libraries like
\texttt{wide} and \texttt{std::simd} (which is still in development) that provide abstractions for SIMD programming and could significantly boost the
performance of both serial and parallel implementations.

\vspace{1em}

Another area of interest would be to look at \textbf{distributed computing} approaches for extremely large datasets that don't fit into memory on a
single machine. While this project didn't include MPI, revisiting distributed memory parallelism could provide insight into scaling linear algebra computations beyond a single workstation.

\vspace{1em}

Lastly, deeper numerical accuracy analysis and profiling could be explored. Comparing how different languages and approaches handle
floating-point errors or identifying performance bottlenecks at the instruction level using profiling tools like \texttt{gprof}, \texttt{perf}, or
Rust's \texttt{flamegraph} crate could provide more technical depth and further guide optimization efforts.

\vspace{1em}

In summary, while this project lays a solid foundation for understanding parallelism in matrix operations, there is much room for future work in exploring
additional algorithms, more advanced hardware acceleration techniques, and deeper performance analysis.


% Begin Bibliography
\newpage
\begin{thebibliography}{9}
    \bibitem{openmpSpec2020}
    OpenMP Architecture Review Board, \textit{OpenMP Application Programming Interface Version 5.1},
    \url{https://www.openmp.org/specifications/}, 2020.

    \bibitem{rayonGitHub}
    Niko Matsakis and Rayon Contributors, \textit{Rayon: A data parallelism library for Rust},
    \url{https://github.com/rayon-rs/rayon}, Accessed 2025.

    \bibitem{rayonDocs}
    Rayon Contributors, \textit{Rayon Crate Documentation}, \url{https://docs.rs/rayon/latest/rayon/}, Accessed 2025.
\end{thebibliography}

\end{document}
